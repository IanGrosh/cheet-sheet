% Exam 1 cheet sheet
\documentclass[10pt,landscape]{article}
\usepackage{multicol}
\usepackage{calc}
\usepackage{ifthen}
\usepackage[landscape]{geometry}
\usepackage{hyperref}
\usepackage{cheatSheet}
\usepackage{setspace}
\usepackage{forest}
\newcommand{\tabitem}{~~\llap{\textbullet}~~}

\begin{document}
	
	\raggedright
	\footnotesize
	\begin{multicols}{3}
		% multicol parameters
		% These lengths are set only within the two main columns
		%\setlength{\columnseprule}{0.25pt}
		\setlength{\premulticols}{1pt}
		\setlength{\postmulticols}{1pt}
		\setlength{\multicolsep}{1pt}
		\setlength{\columnsep}{2pt}
		
		\begin{center}
			\Large{\textbf{CSE450 Exam Cheat Sheet}} \\
		\end{center}
		
		\section{Regular Expressions}
		\verb|*| Matches the previous element zero or more times.\\
		\verb|+| Matches the previous element one or more times.\\
		\verb|?| Matches the previous element zero or one time.\\
		\verb|{n}| Matches the previous element exactly n times.\\
		\verb|{n,}| Matches the previous element at least n times.\\
		\verb|{n,m}| Matches the previous element at least n times, but no more than m times.\\
		\verb|[character_group]|  Matches any single character in \verb|character_group|. By default, the match is case-sensitive.\\
		\verb|[^character_group]| Negation: Matches any single character that is not in \verb|character_group|. By default, characters in \verb|character_group| are case-sensitive.\\
		\verb|[first-last]|  Character range: Matches any single character in the range from \textit{first} to \textit{last}.\\
		.  Matches any single character in the Unicode general category or named block specified by name.\\
		\verb|^|  The match must start at the beginning of the string or line. \\
		\verb|$|  The match must occur at the end of the string or before \verb|\n| at the end of the line or string.
	
		
		\subsection{Project 5 solution lex}
	
		\begin{tabular}{ l r }
			  \verb|VAL_LITERAL|    & \verb!r'((\d+)(\.\d+)?)|(\.\d+)'! \\
			  \verb|CHAR_LITERAL| & \verb!r"'([^\\']|\\n|\\t|\\'|\\\\)'"! \\
			  \verb|STRING_LITERAL| & \verb!r'"([^\\"]|\\n|\\t|\\"|\\\\)*"'!\\
			  \verb|ID| & \verb|r'[a-zA-Z_][a-zA-Z_0-9]*'|
		\end{tabular} 
		\begin{tabular}{ l r | l r }
			  \verb|ASSIGN_ADD| & \verb|r'\+='| &
			  \verb|ASSIGN_SUB| & \verb|r'\-='| \\
			  \verb|ASSIGN_MULT| & \verb|r'\*='| &
			  \verb|ASSIGN_DIV| & \verb|r'/='| \\
			  \verb|COMP_EQU| & \verb|r'=='| &
			  \verb|COMP_NEQU| & \verb|r'!='| \\
			  \verb|COMP_LTE| & \verb|r'<='| &
			  \verb|COMP_GTE| & \verb|r'>='| \\
			  \verb|COMP_LESS| & \verb|r'<'| &
			  \verb|COMP_GTR| & \verb|r'>'| \\
			  \verb|BOOL_AND| & \verb|r'&&'| &
			  \verb|BOOL_OR| & \verb!r'\|\|'! \\
			  \verb|WHITESPACE| & \verb|r'[ \t]'| &
			  \verb|COMMENT| & \verb|r'\#[^\n]*'| \\
			  \verb|newline| & \verb|r'\n+'| 
		\end{tabular} 
		\section{Context Free Grammars}
        CFGs Consist of 4 components (Backus-Naur Form or BNF): 
%\begin{itemize}[leftmargin=*]
          \begin{tabular}{p{0.23\textwidth}l}

            Terminal Symbols = token or $\epsilon$ & $S \rightarrow a S a$ \\
            Non-terminal Symbols = syntactic variables  & $S \rightarrow  T $\\
            Start Symbol S = special non-terminal  &  $T \rightarrow b S b $ \\
           Production Rules of the form LHS $\rightarrow$ RHS \begin{itemize}
            \item LHS = A single non-terminal 
            \item RHS = A string of terminals and non-terminals
            \item Specify how non-terminals may be expanded
            \item By default, the LHS of the first production rule is the Start Symbol
            \end{itemize} & $ T ➔ \epsilon$
            
          \end{tabular}
%\end{itemize}
Shorthand - vertical bar '\verb!|!' to combine multiple productions \\
$S \rightarrow a S a | T $ \\
$T \rightarrow b T b | \epsilon $\\
		\subsection{project 5 CFG}
        %\begin{spacing}{\onehalfspacing}
		\begin{verbatim}
		program : statements
		\end{verbatim}
		\begin{verbatim}
		statements :
		\end{verbatim}
		\begin{verbatim}
		statements : statements statement
		\end{verbatim}
		\begin{tabular}{ll}
		statement 	&\verb!:! expression ';' \\
			&\verb!|! print\_statement ';'\\
			&\verb!|! declaration ';'\\
			&\verb!|! block\\
			&\verb!|! if\_statement\\
			&\verb!|! while\_statement\\
		\end{tabular}
		\begin{verbatim}
		statement : ';'
		\end{verbatim}
		\begin{verbatim}
		statement : FLOW_BREAK ';'
		\end{verbatim}
		\begin{verbatim}
		if_statement : 
		FLOW_IF '(' expression ')' statement %prec IFX
		\end{verbatim}
		\begin{verbatim}
		if_statement : 
		FLOW_IF '(' expression ')' statement FLOW_ELSE statement
		\end{verbatim}
		\begin{verbatim}
		while_statement : 
		FLOW_WHILE '(' expression ')' statement
		\end{verbatim}
		\begin{verbatim}
		block : '{' new_scope statements '}'
		\end{verbatim}
		\begin{verbatim}
		"new_scope :"
		\end{verbatim}
		\begin{verbatim}
		print_statement : 
		COMMAND_PRINT '(' non_empty_comma_sep_expr ')'
		\end{verbatim}
		\begin{verbatim}
		non_empty_comma_sep_expr : expression
		\end{verbatim}
		\begin{verbatim}
		non_empty_comma_sep_expr :
		non_empty_comma_sep_expr ',' expressi\usepackage{tikz}
\usetikzlibrary{shapes}on
		\end{verbatim}
		\begin{verbatim}
		expression : var_usage '=' expression
		\end{verbatim}
		\begin{verbatim}
		expression : expression '+' expression
		| expression '-' expression
		| expression '*' expression
		| expression '/' expression
		\end{verbatim}
		\begin{verbatim}
		expression : '-' expression %prec UMINUS
		\end{verbatim}
		\begin{verbatim}
		expression : '!' expression
		\end{verbatim}
		\begin{verbatim}
		expression : var_usage ASSIGN_ADD expression
		| var_usage ASSIGN_SUB expression
		| var_usage ASSIGN_DIV expression
		| var_usage ASSIGN_MULT expression
		\end{verbatim}
		\begin{verbatim}
		expression : expression COMP_EQU expression
		| expression COMP_NEQU expression
		| expression COMP_LTE expression
		| expression COMP_LESS expression
		| expression COMP_GTR expression
		| expression COMP_GTE expression
		\end{verbatim}
		\begin{verbatim}
		expression : 
		expression BOOL_AND expression
		| expression BOOL_OR expression
		\end{verbatim}
		\begin{verbatim}
		simple_declaration : type ID
		\end{verbatim}
		\begin{verbatim}
		assign_declaration : simple_declaration '=' expression
		\end{verbatim}
		\begin{verbatim}
		expression : ID '.' ID '(' ')'
		\end{verbatim}
		\begin{verbatim}
		statement : ID '.' ID '(' expression ')'
		\end{verbatim}
		\begin{verbatim}
		declaration : simple_declaration
		| assign_declaration
		\end{verbatim}
		\begin{verbatim}
		var_usage : ID
		\end{verbatim}
		\begin{verbatim}
		expression : var_usage
		\end{verbatim}
		\begin{verbatim}
		expression : STRING_LITERAL
		\end{verbatim}
		\begin{verbatim}
		expression : CHAR_LITERAL
		\end{verbatim}
		\begin{verbatim}
		expression : '(' expression ')'
		\end{verbatim}
		\begin{verbatim}
		type : ARRAY_KEYWORD '(' TYPE ')'
		\end{verbatim}
		\begin{verbatim}
		var_usage : ID '[' expression ']'
		\end{verbatim}
		\begin{verbatim}
		type : STRING_KEYWORD
		\end{verbatim}
		\begin{verbatim}
			expression : COMMAND_RANDOM '(' expression ')'
		\end{verbatim}
		
		%\end{spacing}
        \section{Tube IC}
        Scaler ones:
        \begin{tabular}{p{0.1\textwidth} p{0.23\textwidth}}
        \verb!val_copy! s1 s2       & s2 = s1 \\
        \verb!add! s1 s2  s3        & s3 = s1 + s2\\
        \verb!sub! s1 s2  s3        & s3 = s1 - s2 \\
        \verb!mult! s1 s2  s3       & s3 = s1 * s2 \\
        \verb!div! s1 s2  s3        & s3 = s1 / s2 \\
        \verb!test_less! s1 s2  s3  & If (s1 $<$ s2) set s3 to 1, else set s3 to 0. \\
        \verb!test_gtr! s1 s2  s3   & If (s1 $>$ s2) set s3 to 1, else set s3 to 0. \\
        \verb!test_equ! s1 s2  s3   & If (s1 $==$ s2) set s3 to 1, else set s3 to 0. \\
        \verb!test_nequ! s1 s2  s3  & If (s1 $!=$ s2) set s3 to 1, else set s3 to 0. \\
        \verb!test_gte! s1 s2  s3   & If (s1 $>=$ s2) set s3 to 1, else set s3 to 0. \\
        \verb!test_lte! s1 s2  s3   & If (s1 $<=$ s2) set s3 to 1, else set s3 to 0. \\
        \verb!jump! Lable       & jump to the lable \\
        \verb!jump_if_0! s1 Lable      & If s1 == 0, jump to Lable. \\
        \verb!jump_if_n0! s1 Lable       & If s1 != 0, jump to Lable. \\
        \verb!random! s1 s2       & s2 = a random integer x, where 0 $<=$ x $<$ s1. \\
        \verb!out_val! s1       & Write a floating-point value of s1 to standard out. \\
        \verb!out_char! s1      & Write s1 as charto standard out. \\
        \end{tabular}
        array ones:
        \begin{tabular}{p{0.1\textwidth} p{0.2\textwidth}}
        \verb!ar_get_idx! a1 s2 s3       & In a1, find value at index s2, and put into s1. \\
        \verb!ar_set_idx! a1 s2 s3        & In a1, set value at index s2 to the value s3\\
        \verb!ar_get_size! a1 s2        & Calculate the size of a1 and put into s2. \\
        \verb!ar_set_size! a1 s2       & Resize a1 to have s2 entries. \\
        \verb!ar_copy! a1 a2       & Duplicate all values within a1 into a2.\\
        \end{tabular}

        \subsection{Tube AC}
        \begin{itemize}
        \begin{tabular}{p{0.25\textwidth}}
        \item There are no scalar or array variables. \\
        \item There are eight registers called regA, regB, regC, regD, regE, regF, regG, and regH. These are identical to scalar variables, but you have a limited number of them.\\
        \item There are no array-based instructions so you must find replacements for the array instructions.\\   
        \end{tabular}
        \end{itemize}

		
        \section{Flow Control exampls}
        using them jumps
        
        \subsection{IF example}
        \begin{minipage}{0.1\textwidth}
        
        \begin{verbatim}
val x = 1;
if(x) {
    print(x);
}
        \end{verbatim}
                \textbf{IC:}\\
        \verb!val_copy! 1 s2 \\
        \verb!val_copy! s2 s1  \\
        \verb!jump_if_0! s1 \verb!if_1! \\
        \verb!out_val! s1 \\
        \verb!out_char! '$\backslash $n' \\
        \verb!if_1:!
    \end{minipage}
    \begin{minipage}{0.2\textwidth}
    \begin{center}
    \begin{forest}
      for tree={l+=1em} % increase level distance
      [program
        [ asine 
            [declare 
                [val] 
                [x]
            ]
            [1]
        ]
        [if 
            [x]
            [block 
                [print 
                    [x]
                ]
            ]
        ]
      ]
    \end{forest}
    \end{center}
    \end{minipage}%

    \subsection{WHILE}
\begin{minipage}{0.1\textwidth}
            \begin{verbatim}
val x = 6;
val y = 0;
while(y < x) {
    y += 1;
}
        \end{verbatim}
                \textbf{IC:}\\
        \verb!val_copy! 6 s2 \\
        \verb!val_copy! s2 s1  \\
        \verb!val_copy! 0 s4 \\
        \verb!val_copy! s4 s3  \\
        \verb!start_1:! \\
        \verb!test_less! s3 s1 s5 \\
        \verb!jump_if_0! s5 \verb!end_2! \\
        \verb!val_copy! 1 s6 \\
        \verb!add! s3 s6 s7 \\ 
        \verb!val_copy! s7 s3 \\
        \verb!jump! \verb!start_1! \\
        \verb!end_2:!
    \end{minipage}%
    \begin{minipage}{0.2\textwidth}
    \begin{center}
    \begin{forest}
      for tree={l+=1em} % increase level distance
      [program
        [ asine 
            [declare 
                [val] 
                [x]
            ]
            [6]
        ]
        [ asine 
            [declare 
                [val] 
                [y]
            ]
            [0]
        ]
        [while 
            [$<$ [y] [x] ]
            [block 
                [asine 
                    [y]
                    [$+$ [y] [1] ]
                ]
            ]
        ]
      ]
    \end{forest}
    \end{center}
    \end{minipage}

	\end{multicols}
\end{document}
